\nonstopmode
\documentclass[t,fleqn,usenames,dvipsnames]{beamer}
%\documentclass{article}\usepackage{beamerarticle}

\makeatletter
%\def\leqn{\tagsleft@true}
%\def\reqn{\tagsleft@false}
\def\fleq{\@fleqntrue\let\mathindent\@mathmargin \@mathmargin=\leftmargini}
\def\cneq{\@fleqnfalse}
\g@addto@macro{\endsubequations}{\addtocounter{equation}{-1}}
\makeatother

\mode<presentation>
{
  % \usetheme{AnnArbor}
  \useinnertheme[shadow=true]{rounded}
  \useoutertheme{infolines}
  %\setbeamercolor*{title in head/foot}{parent=palette secondary}
  %\useoutertheme{shadow}
  \usecolortheme{wolverine}

  \setbeamerfont{block title}{size={}}
  \setbeamercolor{titlelike}{parent=structure,bg=yellow!85!orange}

  % My modification: centered title with white background
  \setbeamercolor{frametitle}{bg=white}
  \setbeamertemplate{frametitle}{
    %\begin{center}
      \Large\insertframetitle
      \vspace{1ex}
      %\hline
      \par
    %\end{center}
  }
  % %% Does not work
  % \setbeamercolor{frametitlecentered}{bg=white}
  % \setbeamertemplate{frametitlecentered}{
  %   \begin{center}
  %     \Large\insertframetitle
  %     \par
  %   \end{center}
  % }
  % plus red emphasis
  \setbeamercolor{alerted text}{fg=red!70!black}

  % More space between items
  %\defbeamertemplate*{itemize/enumerate body begin}{default}{\itemsep1ex}


  \setbeamercolor{math text}{parent=titlelike}
%  \setbeamercolor{math text displayed}{parent=palette primary}

  % Fix background of theorems/proof
  \setbeamercolor{block body}{parent=palette primary}%{bg=yellow!85!orange}
  \setbeamercolor{block title}{parent=palette secondary}%{bg=orange}



  \setbeamercovered{transparent}
  % or whatever (possibly just delete it)

} % end mode presentation

%\usepackage{tcolorbox}
\usepackage[english]{babel}
\usepackage[utf8x]{inputenc}
%\usepackage{times}  % DYSFUNCTIONAL! looks awful when mixing mathtt and ordinary math.
\usepackage{ifthen}
\usepackage{xspace}
%\usepackage[T1]{fontenc} %no effect
% Or whatever. Note that the encoding and the font should match. If T1
% does not look nice, try deleting the line with the fontenc.
\usepackage{bbm}
\usepackage{amsmath}
\usepackage{amssymb}
\usepackage{stmaryrd}
\usepackage{alltt}
\usepackage[normalem]{ulem} % strikethrough \sout
\usepackage{cancel} % strikethrough math mode \cancel
%\usepackage{enumitem} % set label in itemize
\usepackage{pifont} % for \tickNo
%\usepackage{}
%\usepackage{listings}
\usepackage[all]{xy}
%\usepackage{proof}
%\usepackage{eurosym}
%\usepackage{graphics}
\usepackage{bibentry}
%\nobibliography{short}
\bibliographystyle{plain}
%\input{prooftree}

\newtheorem{takehomemessage}[theorem]{\textbf{Take home message}}

%\usepackage{agda}
%\AgdaNoSpaceAroundCode{}

\DeclareMathSymbol{\chkmark}{\mathord}{AMSa}{"58}

% RGB colors
\definecolor{darkred}{rgb}{0.5,0,0}
\definecolor{darkgreen}{rgb}{0,0.5,0}
\definecolor{darkblue}{rgb}{0,0,0.5}
\definecolor{dirtyred}{rgb}{0.7,0.2,0.1}
\definecolor{dirtygreen}{rgb}{0.2,0.4,0.1}
\definecolor{darkdirtygreen}{rgb}{0.13,0.25,0.07}
\definecolor{dirtyblue}{rgb}{0.07,0.2,0.5}
\definecolor{darkdirtyblue}{rgb}{0.1,0.15,0.35}
\definecolor{lightblue}{rgb}{0.5,0.5,1}
\definecolor{olivegreen}{rgb}{0.5,0.5,0}
\definecolor{brown}{rgb}{0.65,0.35,0} % almost gold
\definecolor{grey}{rgb}{0.33,0.33,0.33}
\definecolor{darkbrown}{rgb}{0.35,0.15,0}
\definecolor{darkgrey}{rgb}{0.16,0.16,0.16}

\newcommand{\textred}[1]{{\color{dirtyred}\textbf{#1}}}

%\newcommand{\tickYes}{{\color{darkgreen}\checkmark}} % does not work
  %since checkmark invokes math mode
\newcommand{\tickYes}{\ensuremath{\color{darkgreen}\chkmark}}
\newcommand{\tickNo}{{\color{dirtyred}\hspace{1pt}\ding{55}}}

%\usepackage[curve,matrix,arrow]{xy}
%\usepackage{tikz} % Drawing diagrams
%\usepackage{pgflibraryshapes} %Ellipses


% MACROS:
\newcommand{\inst}{}
\RequirePackage{ifthen}
%% NOT FOUND: \RequirePackage{mnsymbol} % \dotminus
%\RequirePackage{amssymb}
%\RequirePackage{amsfonts}
\RequirePackage{stmaryrd} % \shortuparrow
\RequirePackage{amssymb}

% evergreens
\newcommand{\bla}{\ensuremath{\mbox{$$}}} % invisible, but not ignored
\newcommand{\Fpure}{\ensuremath{\mathsf{F}}}
\newcommand{\Fomega}{\ensuremath{\mathsf{F}^\omega}}
\newcommand{\der}{\,\vdash}
\newcommand{\derP}{\der}%^+}
\newcommand{\of}{\!:\!}
\newcommand{\is}{\!=\!}
\newcommand{\sub}{\!\leq\!}
\newcommand{\red}{\longrightarrow}
\newcommand{\cored}{\longleftarrow}
\newcommand{\coredp}{\mathop{{}^+{\longleftarrow}}\nolimits}
\newcommand{\redbeta}{\red_\beta}
\newcommand{\redbetastar}{\red_\beta^*}
\newcommand{\coredbetastar}{\mathrel{\bla_\beta^*{\longleftarrow}}}
\newcommand{\redeta}{\red_\eta}
\newcommand{\expeta}{\red_{\eta^-}}
%\newcommand{\redbeta}{\mathrel{\red_\beta}}
%\newcommand{\redeta}{\mathrel{\red_\eta}}
%\newcommand{\coredeta}{\longleftarrow_\eta}
\newcommand{\redetastar}{\red_\eta^*}
\newcommand{\coredetastar}{\mathrel{\bla_\eta^*{\longleftarrow}}}
\newcommand{\redbetaeta}{\red_{\beta\eta}}
\newcommand{\redbetaetastar}{\red_{\beta\eta}^*}
\newcommand{\redw}{\mathrel{\red_{\mathsf{w}}}}
\newcommand{\whd}{\searrow} % for algorithmic equality
\newcommand{\rk}{\mathop{\mathsf{rk}}\nolimits}
\newcommand{\restrict}{\upharpoonright}
\newcommand{\isdef}{\downharpoonright}
\newcommand{\FV}{\ensuremath{\mathsf{FV}}}
\newcommand{\arity}{\ensuremath{\mathsf{ar}}}
%\newcommand{\otto}{\leftrightarrow} %contained in nath.sty
\newcommand{\To}{\Rightarrow}
\newcommand{\Pot}{\mathcal{P}}
\newcommand{\dom}[1]{\mathsf{dom}(#1)}
\newcommand{\rng}{\mathop{\mathsf{rng}}\nolimits}
\newcommand{\pto}{\rightharpoonup} % partial map
\newcommand{\perto}{\stackrel{\mathsf{p}}\rightharpoonup} % partial coherent map
\newcommand{\pertot}{\stackrel{\mathsf{p}}\to} % coherent map
\newcommand{\kto}{\stackrel{\mathsf{K}}\rightharpoonup} % partial Kripke map
\newcommand{\NN}{\mathbb{N}}
\newcommand{\defas}{\mathrel{\ :\Longleftrightarrow\ }}
\newcommand{\defiff}{\mathrel{:\Longleftrightarrow}}
\newcommand{\substo}{\!\mapsto\!}
\newcommand{\mapstot}{\mathrel{\dot\mapsto}}
\newcommand{\condeq}{\doteq}
\newcommand{\aeq}{=_\alpha}
\newcommand{\beq}{=_{\beta}}
\newcommand{\beeq}{=_{\beta\eta}}
\newcommand{\enc}[1]{\ulcorner#1\urcorner}
%% for palatino
%\def\lv{\mathopen{[\kern-0.2em[}}    % opening [[ value delimiter
%\def\rv{\mathclose{]\kern-0.2em]}}   % closing ]] value delimiter
% for computer modern
\def\lv{\mathopen{{[\kern-0.14em[}}}    % opening [[ value delimiter
\def\rv{\mathclose{{]\kern-0.14em]}}}   % closing ]] value delimiter
\newcommand{\den}[1]{\lv #1 \rv}
\newcommand{\Den}[2]{\den{#1}_{#2}}
\newcommand{\dent}[2]{\llparenthesis#1\rrparenthesis_{#2}}
\newcommand{\denty}[1]{\llparenthesis#1\rrparenthesis}
  % banana brackets, need stmaryrd.sty
% for computer modern
% den in superscript
\newcommand{\dens}[1]{\mathopen{[\kern-0.3ex[}#1\mathclose{]\kern-0.3ex]}}
% kind interp.
\newcommand{\denk}[2]{\mathopen{\{\kern-0.3ex|}#1\mathclose{|\kern-0.3ex\}}_{#2}}
\newcommand{\length}[1]{\|#1\|}

% von Buchholz
\def\tox#1{\buildrel#1\over\to}
\def\ltox#1{\buildrel\raise1pt\hbox{$\scriptstyle#1$}\over\longrightarrow}
\def\toid{\tox{\scriptscriptstyle\rm id}}

\def\tocolow{\buildrel\raise-5pt\hbox{$\scriptscriptstyle+$}\over\rightarrow}

% \dotminus from http://tex.stackexchange.com/questions/114188/special-character-dot-over-dash
\makeatletter
\newcommand{\dotminus}{\mathbin{\text{\@dotminus}}}
\newcommand{\@dotminus}{%
  \ooalign{\hidewidth\raise1ex\hbox{.}\hidewidth\cr$\m@th-$\cr}%
}
\makeatother

% type equality
\def\lo{\mathopen{{\lceil\kern-0.25em\lceil}}}
\def\ro{\mathclose{{\rfloor\kern-0.25em\rfloor}}}
\newcommand{\ten}[1]{\lo#1\ro}
\newcommand{\Ten}[2]{\lo#1\ro_{#2}}
%\newcommand{\Tef}[2]{[#1]_{#2}}
\newcommand{\Tef}[2]{#1#2 \in \TYPE}
\newcommand{\Teq}[4]{#1#2 = #3#4 \in \TYPE}
%\newcommand{\Teq}[4]{\Tef{#1}{#2} = \Tef{#3}{#4}}

% latin etc. abbrev
\newcommand{\abbrev}[1]{#1} % alternative: \emph{#1}
\newcommand{\cf}{\abbrev{cf.}\ }
\newcommand{\eg}{\abbrev{e.\,g.}}
\newcommand{\Eg}{\abbrev{E.\,g.}}
\newcommand{\ie}{\abbrev{i.\,e.}}
\newcommand{\Ie}{\abbrev{I.\,e.}}
\newcommand{\etal}{\abbrev{et.\,al.}}
\newcommand{\wwlog}{w.\,l.\,o.\,g.} % \wlog is ``write into log file''
\newcommand{\Wlog}{W.\,l.\,o.\,g.}
\newcommand{\wrt}{w.\,r.\,t.}

% paragraphs
\newcommand{\para}[1]{\paragraph*{\it#1}}
\newcommand{\paradot}[1]{\para{#1.}}

% proof by cases
\newenvironment{caselist}{%
  \begin{list}{{\it Case}}{%
    %\setlength{\topsep}{2ex}% DOES NOT SEEM TO WORK
    %\setlength{\itemsep}{2ex}%
    %\setlength{\itemindent}{-2ex}%
  }%
}{\end{list}%
}
\newenvironment{subcaselist}{%
  \begin{list}{{\it Subcase}}{}%
}{\end{list}%
}
\newenvironment{subsubcaselist}{%
  \begin{list}{{\it Subsubcase}}{}%
}{\end{list}%
}

\newcommand{\nextcase}{\item~}

% meta-level logic
\newcommand{\mfor}{\ \mbox{for}\ }
\newcommand{\mforsome}{\ \mbox{for some}\ }
\newcommand{\mthen}{\ \mbox{then}\ }
\newcommand{\mif}{\ \mbox{if}\ }
\newcommand{\miff}{\ \mbox{iff}\ }
\newcommand{\motherwise}{\ \mbox{otherwise}}
\newcommand{\mundefined}{\mbox{undefined}}
\newcommand{\mnot}{\mbox{not}\ }
\newcommand{\mand}{\ \mbox{and}\ }
\newcommand{\mor}{\ \mbox{or}\ }
\newcommand{\mimplies}{\ \mbox{implies}\ }
\newcommand{\mimply}{\ \mbox{imply}\ }
\newcommand{\mforall}{\ \mbox{for all}\ }
\newcommand{\mexists}{\mbox{exists}\ }
\newcommand{\mexist}{\mbox{exist}\ }
\newcommand{\mtrue}{\mbox{true}}
\newcommand{\mwith}{\ \mbox{with}\ }
\newcommand{\mwhere}{\ \mbox{where}\ }
\newcommand{\mholds}{\ \mbox{holds}\ }
\newcommand{\munless}{\ \mbox{unless}\ }
\newcommand{\mboth}{\ \mbox{both}\ }
\newcommand{\msuchthat}{\ \mbox{such that}\ }
% proofs
\newcommand{\msince}{\mbox{since}\ }
\newcommand{\mdef}{\mbox{by def.}}
\newcommand{\mass}{\mbox{assumption}}
\newcommand{\mhyp}{\mbox{by hyp.}}
\newcommand{\mlemma}[1]{\mbox{by Lemma~#1}}
\newcommand{\mih}[1][]{\mbox{by ind.hyp.}#1}
\newcommand{\mgoal}[1][]{\mbox{goal\ifthenelse{\equal{#1}{}}{}{~#1}}}
\newcommand{\mby}[1]{\mbox{by #1}}
\newcommand{\minfrule}{\mbox{by inference rule}}


% Inference rules
\newcommand{\rulename}[1]{\ensuremath{\mbox{\sc#1}}}
\newcommand{\rul}[2]{\dfrac{\begin{array}[b]{@{}l@{}} #1 \end{array}}{#2}}
\newcommand{\ru}[2]{\dfrac{\begin{array}[b]{@{}c@{}} #1 \end{array}}{#2}}
\newcommand{\rux}[3]{\ru{#1}{#2}\ #3}
\newcommand{\noru}[3]{\ru{#2}{#3}}
\newcommand{\nru}[3]{#1\ \ru{#2}{#3}}
\newcommand{\nrux}[4]{#1\ \ru{#2}{#3}\ #4}
\newcommand{\dstack}[2]{\begin{array}[b]{c}#1\\#2\end{array}}
\newcommand{\ndru}[4]{#1\ \ru{\dstack{#2}{#3}}{#4}}
\newcommand{\ndrux}[5]{#1\ \ru{\dstack{#2}{#3}}{#4}\ #5}
\newcommand{\lcol}[1]{\multicolumn{1}{@{}l@{}}{{#1}}}
\newcommand{\rcol}[1]{\multicolumn{1}{@{}r@{}}{{#1}}}

% Substitution and function update
% read ``\subst F X A'' as ``substitute F for X in A''
%\newcommand{\subst}[3]{#3[#2 := #1]}
%\newcommand{\subst}[3]{[#1/#2]#3}
\newcommand{\subst}[3]{#3[#1/#2]}
% read ``\update \theta X \G'' as update \theta at point X by \G
\newcommand{\update}[3]{#1[#2 \mapsto #3]}
%\newcommand{\update}[3]{#1,#2 \is #3}


% %% ---------- %< ----------
% \DeclareRobustCommand*{\MakeCapital}[1]{%
% \begingroup
% \def\i{I}\def\j{J}%
% \def\do##1##2{\let##1##2\do}%
% \expandafter\do\@uclclist\relax{\relax\@gobble}%
% \let\protect\@gobble
% \def\do##1{\def##1####1{{\protect##1####1}}\do}%
% \expandafter\do\@caplist\do\relax
% \let\protect\noexpand
% \def\t##1##2{{\protect\t##1##2}}%
% \def\IeC##1{{\protect##1}}% -> inputenc
% \def\@tabacckludge##1##2{{% -> inputenc
% \expandafter\protect\csname\string##1\endcsname##2}}%
% \ifnum\catcode`\"=\active % -> babel
% \lccode`\~=`\"%
% \lowercase{\def~##1{{\protect\"##1}}}%
% \fi
% \edef\@tempa{#1}%
% \expandafter\endgroup
% \expandafter\@capitalize\@tempa}

% \def\@caplist{\"\`\^\~\"\=\.\H\b\c\d\r\u\v}
% \def\@capitalize#1{\uppercase{#1}}
% %% ---------- %< ----------

\newcommand{\evalsto}{\searrow}

\newcommand{\funS}[2]{(#1 : #2) \to}
\newcommand{\allS}[2]{\forall #1 : #2.}
\newcommand{\sigT}[2]{\Sigma #1 \of #2.\,}
\newcommand{\funT}[2]{(#1 \of #2) \to}
\newcommand{\sfunT}[2]{(#1 : #2) \to}
\newcommand{\allT}[2]{\forall #1 \of #2. \,}
%\newcommand{\allT}[2]{[#1 \of #2] \to}

\newcommand{\knet}[3]{\kne{#1,#2 \reifiesto #3}}
\newcommand{\knft}[3]{\knf{#1,#2 \reifiesto #3}}
%\newcommand{\Rbase}[5]{#1 : #2 \reifiesto #3 \der #4 : #5}
%\newcommand{\Rne}[4]{#1 \der #2 \reifiesto #3 : #4}

% macros for composing syntax
\newcommand{\sarr}[3]{\tycol{#2} \tox{\qcol{#1}} \tycol{#3}}
\newcommand{\arr}[3]{#2 \tox{#1} #3}
%\newcommand{\arr}[3]{#2 \mathrel{{}^{#1}{\to}} #3}
\newcommand{\PiT}[4]{\Pi^{#1,#2}#3\,#4}
\newcommand{\FunT}[3]{\Fun\,#2\,\lambda#1 #3}%{\Fun\,#2\,(\lambda#1.#3)}
\newcommand{\FunV}[2]{\Fun\,#1\,#2}
\newcommand{\succT}[1]{\ts\,#1}
\newcommand{\recT}[1][]{\trec\,(#1)} %{\trec_{#1}}
\newcommand{\RecT}[1][]{\Rec\,(#1)} %{\Rec_{#1}}
\newcommand{\lam}[1]{\lambda^{#1}\,}
\newcommand{\nlam}[2]{\lambda^{#2} #1.\,}
\newcommand{\snlam}[2]{\lambda^{\qcol{#2}}\varcol{#1}.\,}
%\newcommand{\nlam}[2]{\lambda #1^{#2}.\,}
\newcommand{\lamdot}[1]{\lambda^{#1}.\,}

\newcommand{\tif}{\mathsf{if}}
\newcommand{\tifzero}{\tmc{ifzero}}
\newcommand{\tthen}{\tmc{then}}
\newcommand{\telse}{\tmc{else}}
\newcommand{\tifthenelse}[3]{\tif\,#1\,\tthen\,#2\,\telse\,#3}
\newcommand{\ifzero}[3]{\tifzero\,#1\,\tthen\,#2\,\telse\,#3}
%\newcommand{\tifthenelse}[3]{\tif\;#1\;\tthen\;#2\;\telse\;#3}

% syntactic classes
\newcommand{\Quant}{\mathsf{Quant}}
\newcommand{\Sort}{\mathsf{Sort}}
\newcommand{\tkind}{\Box}
\newcommand{\Base}{\mathsf{Base}}
\newcommand{\Ty}{\mathsf{Ty}}
\newcommand{\Ann}{\mathsf{Ann}}
\newcommand{\LamII}{\Lambda^{\downarrow\uparrow}}
\newcommand{\Kind}{\mathsf{Kind}}
\newcommand{\Ki}{\mathsf{Ki}}
\newcommand{\Constr}{\mathsf{Constr}}
\newcommand{\Obj}{\mathsf{Obj}}
\newcommand{\Elim}{\mathsf{Elim}}
\newcommand{\Atom}{\mathsf{Atom}}
\newcommand{\Const}{\mathsf{Const}}
\newcommand{\Proj}{\mathsf{Proj}}
\newcommand{\Par}{\mathsf{Par}}
\newcommand{\Clos}{\mathsf{Clos}}
\newcommand{\Cxt}{\mathsf{Cxt}}
\newcommand{\WCxt}{\mathsf{WCxt}}
\newcommand{\WNe}{\mathsf{WNe}}
\newcommand{\Wne}{\mathsf{Wne}}
\newcommand{\Whnf}{\mathsf{Whnf}}
\newcommand{\Wbase}{\mathsf{Wbase}}
\newcommand{\WVal}{\mathsf{WVal}}
\newcommand{\WFun}{\mathsf{WFun}}
\newcommand{\WPair}{\mathsf{WPair}}
\newcommand{\WElim}{\mathsf{WElim}}
\newcommand{\Env}{\mathsf{Env}}
\newcommand{\tTm}{\mathsf{Tm}}
\newcommand{\BVar}{\mathsf{BVar}}
\newcommand{\FVar}{\mathsf{FVar}}
\newcommand{\tNe}{\mathsf{Ne}}
\newcommand{\Ne}[2]{\tNe\;#1\;#2}
\newcommand{\NeExp}{\mathsf{NeExp}}
\newcommand{\Null}{\emptyset}
\newcommand{\Unit}{()}
\newcommand{\tNf}{\mathsf{Nf}}
\newcommand{\Neut}{\mathsf{Neut}}
\newcommand{\Norm}{\mathsf{Norm}}
\newcommand{\VNe}{\mathsf{VNe}}
\newcommand{\VFun}{\mathsf{VFun}}
\newcommand{\VPair}{\mathsf{VPair}}
\newcommand{\FVal}{\mathsf{FVal}}
\newcommand{\PVal}{\mathsf{PVal}}
\newcommand{\GVal}{\mathsf{GVal}}
\newcommand{\tfst}{\mathsf{fst}}
\newcommand{\tsnd}{\mathsf{snd}}
\newcommand{\tS}{\mathsf{S}}
\newcommand{\tK}{\mathsf{K}}
\newcommand{\tI}{\mathsf{I}}

% Greek
\newcommand{\Ge}{\ensuremath{\varepsilon}}
\newcommand{\GG}{\ensuremath{\Gamma}}
\newcommand{\GD}{\ensuremath{\Delta}}
\newcommand{\GL}{\ensuremath{\Lambda}}




% inference and checking
\newcommand{\jne}{\Downarrow}
\newcommand{\jnf}{\Uparrow}
%\newcommand{\jinf}{\Downarrow}
%\newcommand{\jchk}{\Uparrow}
\newcommand{\jinf}{\rightrightarrows}
\newcommand{\jchk}{\leftleftarrows}
%\newcommand{\jinf}{\mathrel{\mathord{:}\mathord{\Rightarrow}}}
%\newcommand{\jchk}{\mathrel{\mathord{:}\mathord{\Leftarrow}}}
\newcommand{\ders}{\der_\mathsf{s}}


% per model
\newcommand{\perpref}{per-}
\newcommand{\rper}[1]{\rulename{\perpref{}#1}}

% % projections
% \newcommand{\tL}{\mathsf{L}}
% \newcommand{\tR}{\mathsf{R}}

% % saturation interval
% \renewcommand{\N}{\mathcal{N}}
% \newcommand{\Sfun}{\mathcal{S}_{\mathit{fun}}}
% \newcommand{\Spair}{\mathcal{S}_{\mathit{pair}}}


% identitys
%\newcommand{\tid}{\mathsf{i\kern-0.1ex d}} % not such a good idea
\newcommand{\tid}{\mathsf{id}}
\newcommand{\tId}{\mathsf{Id}}
% identity substitution
%\newcommand{\sid}{\sigma_{\tid}}

% contexts
%\newcommand{\cempty}{\mathord{\triangledown}}
%\newcommand{\cempty}{\mathord{\diamond}}
%\newcommand{\cempty}{()}
\newcommand{\cempty}{\emptyset}
\newcommand{\cxt}{\ \mathsf{cxt}}

% calligraphic capital letters
\newcommand{\A}{\tycol{A}}
\newcommand{\B}{\tycol{B}}
%\newcommand{\C}{\mathcal{C}}
\newcommand{\DD}{\mathcal{D}}
\newcommand{\E}{\mathcal{E}}
\newcommand{\F}{\mathcal{F}}
%\newcommand{\G}{\mathcal{G}}
\renewcommand{\P}{\mathcal{P}}

% universes (semantically)
\newcommand{\UU}{\mathcal{U}}
\newcommand{\TT}{\TYPE}%{\mathcal{T}}

% term model
\newcommand{\betaeq}{=_\beta}
\newcommand{\etaeq}{=_\eta}

\newcommand{\tleft}{\mathsf{left}}
\newcommand{\tright}{\mathsf{right}}

% misc stuff
\newcommand{\sat}{\mathop{\mathsf{sat}}\nolimits}
\newcommand{\SN}{\mathsf{SN}}
\newcommand{\WN}{\mathsf{WN}}
\newcommand{\TmVar}{\mathsf{TmVar}}
\newcommand{\TyVar}{\mathsf{TyVar}}
\newcommand{\Subst}{\mathsf{Subst}}
\newcommand{\Eq}{\mathsf{Eq}} % for talkProgLog
\newcommand{\List}{\mathsf{List}}
\newcommand{\UList}{\mathsf{UList}}
\newcommand{\tappend}{\mathsf{append}}
\newcommand{\tvcons}{\mathsf{vcons}}
\newcommand{\tvnil}{\mathsf{vnil}}
\newcommand{\vxs}{\mathit{xs}}
\newcommand{\vys}{\mathit{ys}}
\newcommand{\vzs}{\mathit{zs}}
\newcommand{\vas}{\mathit{as}}
\newcommand{\vbs}{\mathit{bs}}
\newcommand{\vcs}{\mathit{cs}}


%%%%%%%%%%%%%%%%%%%%%%%%%%%%%%%%%%%%%%%%%%%%%%%%%%%%%%%%%%%%%%%%%%%%%%
%%%%%%%%%%%%%%%%%%%%%%%%%%%% SemEq stuff %%%%%%%%%%%%%%%%%%%%%%%%%%%%%
%%%%%%%%%%%%%%%%%%%%%%%%%%%%%%%%%%%%%%%%%%%%%%%%%%%%%%%%%%%%%%%%%%%%%%


% \renewcommand{\update}[3]{(#1,#2 \is #3)}
% \renewcommand{\subst}[3]{#3(\sid,#2 \is #1)}

% \renewcommand{\Den}[2]{#1 #2}
% \renewcommand{\den}[1]{#1\rid}

\newcommand{\comp}{\circ}

% de Bruijn index
\newcommand{\ind}[1]{\mathsf{v}_{#1}}
% de Bruijn level
\newcommand{\lev}[1]{\x_{#1}}
\newcommand{\var}[1]{\x_{#1}}
\newcommand{\xdel}[1][]{\ifthenelse{\equal{#1}{}}{\x_\Delta}{\x_{\Delta+#1}}}

\newcommand{\Bool}{\mathsf{Bool}}
\newcommand{\ttrue}{\mathsf{true}}
\newcommand{\tfalse}{\mathsf{false}}


% tables and figures
% deftable, deffigure, figureone, figuretwo, figuretwosqueeze

%Tables
\newenvironment{deftable}[2]{%
  \def\deftablecaption{#2}%
  \begin{table}[htbp]%
  \begin{center}%
  \begin{scriptsize}%
  \begin{minipage}{#1}%
  \hrule \vspace*{4ex}%
%  % reassign dimensions for prooftrees
%  \proofrulebaseline=2ex%
%  \proofrulebreadth=.05em%
%  \proofdotseparation=1.25ex%
}{%
\vspace{2ex} \hrule%
\addvspace{2ex}%
  \end{minipage}%
  \end{scriptsize}%
  \end{center}%
  \caption{\deftablecaption}%
  \end{table}%
}

%Figures
\newenvironment{deffigure}[2]{%
  \def\deffigurecaption{#2}%
  \begin{figure*}[htbp]%
  \begin{center}%
  %\begin{scriptsize}%
  \begin{minipage}{#1}%
  \hrule \vspace*{4ex}%
%  % reassign dimensions for prooftrees
%  \proofrulebaseline=2ex%
%  \proofrulebreadth=.05em%
%  \proofdotseparation=1.25ex%
}{%
\vspace{2ex} \hrule%
\addvspace{2ex}%
  \end{minipage}%
  %\end{scriptsize}%
  \end{center}%
  \caption{\deffigurecaption}%
  \end{figure*}%
}

\newenvironment{figureone}[1]{%
  \def\deffigurecaption{#1}%
  \begin{figure}[htbp]%
  \begin{center}%
  %\begin{scriptsize}%
  \begin{minipage}{\columnwidth}%
  \hrule \vspace*{2ex}%
  % reassign dimensions for prooftrees
  \proofrulebaseline=2ex%
  \proofrulebreadth=.05em%
  \proofdotseparation=1.25ex%
}{%
\vspace{2ex} \hrule%
%\addvspace{2ex}%
  \end{minipage}%
  %\end{scriptsize}%
  \end{center}%
  \caption{\deffigurecaption}%
  \end{figure}%
}

\newenvironment{figuretwo}[1]{%
  \def\deffigurecaption{#1}%
  \begin{figure*}[htbp]%
  \begin{center}%
  %\begin{scriptsize}%
  \begin{minipage}{\textwidth}%
  \hrule \vspace*{2ex}%
  % reassign dimensions for prooftrees
  \proofrulebaseline=2ex%
  \proofrulebreadth=.05em%
  \proofdotseparation=1.25ex%
}{%
%\vspace{2ex}
  \hrule%
%\addvspace{2ex}%
  \end{minipage}%
  %\end{scriptsize}%
  \end{center}%
  \caption{\deffigurecaption}%
  \end{figure*}%
}

\newenvironment{figuretwosqueeze}[1]{%
  \def\deffigurecaption{#1}%
  \begin{figure*}[htbp]%
  \begin{center}%
  %\begin{scriptsize}%
  \begin{minipage}{\textwidth}%
  \hrule \vspace*{2ex}%
  % reassign dimensions for prooftrees
  \proofrulebaseline=2ex%
  \proofrulebreadth=.05em%
  \proofdotseparation=1.25ex%
}{%
%\vspace{2ex}
\hrule%
%\addvspace{2ex}%
  \end{minipage}%
  %\end{scriptsize}%
  \end{center}%
  \caption{\deffigurecaption}%
  \end{figure*}%
}

%

%%% Local Variables:
%%% mode: latex
%%% TeX-master: "talkAIM27"
%%% End:

\renewcommand{\ru}[2]{\dfrac{\begin{array}[b]{@{}c@{}} #1 \end{array}}{\begin{array}[t]{@{}l@{}} #2 \end{array}}}

% black text in mbox
\newcommand{\mybox}[1]{\mbox{\color{black}#1}}

% meta-level logic
\renewcommand{\mfor}{\ \mybox{for}\ }
\renewcommand{\mforsome}{\ \mybox{for some}\ }
\renewcommand{\mthen}{\ \mybox{then}\ }
\renewcommand{\mif}{\ \mybox{if}\ }
\renewcommand{\miff}{\ \mybox{iff}\ }
\renewcommand{\motherwise}{\ \mybox{otherwise}}
\renewcommand{\mundefined}{\mybox{undefined}}
\renewcommand{\mnot}{\mybox{not}\ }
\renewcommand{\mand}{\ \mybox{and}\ }
\renewcommand{\mor}{\ \mybox{or}\ }
\renewcommand{\mimplies}{\ \mybox{implies}\ }
\renewcommand{\mimply}{\ \mybox{imply}\ }
\renewcommand{\mforall}{\ \mybox{for all}\ }
\renewcommand{\mexists}{\mybox{exists}\ }
\renewcommand{\mexist}{\mybox{exist}\ }
\renewcommand{\mtrue}{\mybox{true}}
\renewcommand{\mwith}{\ \mybox{with}\ }
\renewcommand{\mwhere}{\ \mybox{where}\ }
\renewcommand{\mholds}{\ \mybox{holds}\ }
\renewcommand{\munless}{\ \mybox{unless}\ }
\renewcommand{\mboth}{\ \mybox{both}\ }
\renewcommand{\msuchthat}{\ \mybox{such that}\ }
% proofs
\renewcommand{\msince}{\mybox{since}\ }
\renewcommand{\mdef}{\mybox{by def.}}
\renewcommand{\mass}{\mybox{assumption}}
\renewcommand{\mhyp}{\mybox{by hyp.}}
\renewcommand{\mlemma}[1]{\mybox{by Lemma~#1}}
\renewcommand{\mih}[1][]{\mybox{by ind.hyp.}#1}
\renewcommand{\mgoal}[1][]{\mybox{goal\ifthenelse{\equal{#1}{}}{}{~#1}}}
\renewcommand{\mby}[1]{\mybox{by #1}}
\renewcommand{\minfrule}{\mybox{by inference rule}}
\newcommand{\ONE}{\mathbf{1}}

% DOES NOT WORK:
% rule with "boring" default text
\newcommand{\setboring}{\setbeamercolor{math text}{fg=grey}\setbeamercolor{math display}{fg=grey}}
\newcommand{\bru}[2]{\setboring\dfrac{\setboring#1}{\setboring#2}}

\renewcommand*\ttdefault{txtt} % for listing package

% \newcommand{\defHaskelllistings}{%
%   \lstset{%
%     language=Haskell,%
%     basicstyle=\ttfamily\small\color{darkdirtyblue},% \ttfamily
%     keywordstyle=\ttfamily\bfseries,% \underbar
%     identifierstyle=,%
%     commentstyle=\itshape,%
%     columns=flexible,%spaceflexible,% fixed,% flexible,%
%     showstringspaces=false,%
% %    xleftmargin=\codeindent,% defined below
%     breaklines=true,%
%     deletekeywords={succ,zero,head,tail,zipWith,Either,List},%
%     morekeywords={Set,Size,fun,cofun,pattern},% ,left,right,nil,cons
%     literate={\\}{{$\lambda$}}1 {->}{{$\rightarrow$~}}2
%              {<=}{{$\leq$~}}2 {<}{{$<$~}}1
% %     literate={map}{map~}4
% %       {even}{even~}5
% %       {odd}{odd~}4
%      }%
% }
%\defHaskelllistings

\title[Dependently-Typed Compilation]{%
Type-preserving compilation \\ via dependently typed syntax in Agda}

\author[Abel]{
  Andreas Abel\inst{1}
}
%{F.~Author\inst{1} \and S.~Another\inst{2}}
% - Give the names in the same order as the appear in the paper.
% - Use the \inst{?} command only if the authors have different
%   affiliation.

\institute[] %Chalmers/GU/ENS Cachan] % (optional, but mostly needed)
{
  \inst{1}
  Department of Computer Science and Engineering\\
  Chalmers and Gothenburg University, Sweden \\[1ex]
}
%  \inst{1}%
%  Department of Computer Science\\
%  University of Somewhere
%  \and
%  \inst{2}%
%  Department of Theoretical Philosophy\\
%  University of Elsewhere}
%% - Use the \inst command only if there are several affiliations.
%% - Keep it simple, no one is interested in your street address.

\date[IFIP 1.3 2020] % (optional, should be abbreviation of conference name)
{ IFIP 1.3 Meeting 2020\\
  Massa Marittima \\
  16 January 2020
}
% - Either use conference name or its abbreviation.
% - Not really informative to the audience, more for people (including
%   yourself) who are reading the slides online

%\subject{Software Verification}
% This is only inserted into the PDF information catalog. Can be left
% out.



% If you have a file called "university-logo-filename.xxx", where xxx
% is a graphic format that can be processed by latex or pdflatex,
% resp., then you can add a logo as follows:

% \pgfdeclareimage[height=0.5cm]{university-logo}{university-logo-filename}
% \logo{\pgfuseimage{university-logo}}



% Delete this, if you do not want the table of contents to pop up at
% the beginning of each subsection:

%\AtBeginSubsection[]
%\AtBeginSection[]
%{
%  \begin{frame}<beamer>
%    \frametitle{Outline}
%    \tableofcontents[currentsection,currentsubsection]
%  \end{frame}
%}


% If you wish to uncover everything in a step-wise fashion, uncomment
% the following command:

%\beamerdefaultoverlayspecification{<+->}

\makeatletter
\def\MLine#1{\par\vspace{1ex}\hspace*{-\@totalleftmargin}\parbox{\textwidth}{\[#1\]}}
\makeatother
%\newenvironment{display}{\par\vspace{1ex}\hspace*{-\@totalleftmargin}\parbox{\textwidth}{\[}{\]}}


% COLORS
\newcommand{\cHead}{\color{darkblue}}
\newcommand{\cSub}{\color{brown}}
\newcommand{\cWhite}{\color{white}}
\newcommand{\cGray}{\color{gray}}
\newcommand{\cGreen}{\color{olivegreen}}
\newcommand{\cBrown}{\color{brown}}
\newcommand{\cBlack}{\color{black}}
\newcommand{\black}[1]{{\cBlack#1}}

\newcommand{\cAnn}{\color{red!80!black}}%purple darkblue
\newcommand{\cAside}{\color{gray}}
\newcommand{\cEnum}{\color{darkgreen}}
\newcommand{\cEm}{\cAnn} %\color{red}}
\newcommand{\cCo}{\cAnn} %\color{red}} % copattern color
\newcommand{\cop}[1]{{\cCo#1}}
\newcommand{\cApp}{\color{violet}}
\newcommand{\capp}[1]{{\cApp#1}}
\newcommand{\cFocus}{\color{darkgreen}}
\newcommand{\focus}[1]{{\cFocus#1}}
\newcommand{\cMath}{\usebeamercolor[fg]{math text}}
\newcommand{\cIdent}{\usebeamercolor[fg]{math text}}
\newcommand{\ident}[1]{{\cIdent#1}}
\newcommand{\cExp}{\cIdent}
\newcommand{\cBoring}{\color{grey}}
\newcommand{\boring}[1]{{\cBoring#1}}

%\newcommand{\ann}[1]{^{\cAnn #1}}
\newcommand{\unn}[1]{_{\cAnn #1}}
\newcommand{\annW}[1]{^{\hphantom{#1}}}
%\newcommand{\Ann}[1]{{\cAnn #1}}
\newcommand{\AnnW}[1]{\hphantom{#1}}
\newcommand{\ttAnn}[1]{\{{\cAnn #1}\}}
% ordinal annotation
\newcommand{\cOrd}{\cAnn}
\newcommand{\onn}[1]{^{\cOrd #1}}
\newcommand{\Onn}[1]{{\cOrd #1}}
\newcommand{\oforall}[1]{\forall\Onn{#1}.~}
\newcommand{\oexists}[1]{\exists\Onn{#1}.~}
\newcommand{\oapp}[1]{\,\Onn{#1}}
\renewcommand{\emph}[1]{{\cAnn#1}}
\newcommand{\OSize}{\Onn{Size}}
\newcommand{\oi}{\Onn{i}}
\newcommand{\odi}{\Onn{\$i}}
\newcommand{\oddi}{\Onn{\$\$i}}
\newcommand{\oj}{\Onn{j}}
\newcommand{\ohash}{\Onn{\#}}

% \newcommand{\kw}[1]{{\bf#1}}
\newcommand{\kwdata}{\kw{data}}
\newcommand{\kwcodata}{\kw{codata}}
\newcommand{\kwsized}{\kw{sized}}
\newcommand{\kwfun}{\kw{fun}}
\newcommand{\kwcofun}{\kw{cofun}}
\newcommand{\kwlet}{\kw{let}}
\newcommand{\kwfields}{\kw{fields}}
%\newcommand{\kw}{\kw{}}
%\newcommand{\kw}{\kw{}}
%\newcommand{\kw}{\kw{}}
\newcommand{\tinl}{\mathsf{inl}}
\newcommand{\tinr}{\mathsf{inr}}

\renewcommand{\rulename}[1]{#1}

% types
\newcommand{\cType}{\color{orange!60!black}}


\newcommand{\mlsays}[1]{}
% {\begin{frame}%[fragile=singleslide]
%   %\frametitle{Himself}
%   \begin{minipage}[c]{0.5\linewidth}
%    \includegraphics[height=0.9\textheight]{martin-loef-bw.png}
%   \end{minipage}% NO SPACE HERE!
%   \begin{minipage}[c]{0.5\linewidth}
%    \begin{center}
%      \bla \\[8ex]
%      \Huge #1
%    \end{center}
%   \end{minipage}
% \end{frame}
% }

\newcommand{\JOKE}[1]{} % No jokes.
\renewcommand{\emph}[1]{\textit{\cType#1}}

\newcommand{\ope}{\twoheadrightarrow}

\DeclareMathOperator*{\amp}{\&}
\DeclareMathOperator*{\bigamp}{\scalerel*{\&}{\sum}}
\usepackage{scalerel}

\newcommand{\tSet}{\mathsf{Set}}
\newcommand{\Set}{\mathsf{Set}}

\newcommand{\tHom}{\mathsf{Hom}}

\newcommand{\tVal}{\mathsf{Val}}
\newcommand{\Val}[1][]{\tVal_{#1}}
\newcommand{\tVar}{\mathsf{Var}}
\newcommand{\Var}[2][]{\mathsf{Var}_{#1}\,#2}
\newcommand{\tExp}{\mathsf{Exp}}
\newcommand{\Exp}[2][]{\mathsf{Exp}_{#1}\,#2}
\newcommand{\tStm}{\mathsf{Stm}}
\newcommand{\Stm}[3][]{\mathsf{Stm}_{#1}\,#2\,#3}
\newcommand{\tStms}{\mathsf{Stms}}
\newcommand{\Stms}[3][]{\mathsf{Stms}_{#1}\,#2\,#3}
\newcommand{\vrt}{\mathit{r}}
\newcommand{\vop}{\mathit{op}}
\newcommand{\tlit}{\mathsf{lit}}

\newcommand{\tvar}{\mathsf{var}}
\newcommand{\tarith}{\mathsf{arith}}
\newcommand{\tcmp}{\mathsf{cmp}}
\newcommand{\tCmpOp}{\mathsf{CmpOp}}
\newcommand{\CmpOp}[1]{\tCmpOp\,#1}
\newcommand{\tArithOp}{\mathsf{ArithOp}}
\newcommand{\ArithOp}[1]{\tArithOp\,#1}

\newcommand{\tassign}{\mathsf{assign}}
\newcommand{\twhile}{\mathsf{while}}
\newcommand{\tifElse}{\mathsf{ifElse}}
\newcommand{\tdecl}{\mathsf{decl}}
\newcommand{\treturn}{\mathsf{return}}
\newcommand{\tldc}{\mathsf{ldc}}
\newcommand{\tload}{\mathsf{load}}
\newcommand{\tstore}{\mathsf{store}}
%\newcommand{\tarith}{\mathsf{arith}}


\newcommand{\tskip}{\mathsf{skip}}
\newcommand{\tkeep}{\mathsf{keep}}
\newcommand{\tdone}{\mathsf{done}}
\newcommand{\sublist}{\subseteq}
\newcommand{\tint}{\mathsf{int}}
\newcommand{\tbool}{\mathsf{bool}}
\newcommand{\tvoid}{\mathsf{void}}
\newcommand{\tdouble}{\mathsf{double}}

\newcommand{\ST}{\mathsf{ST}}
\newcommand{\MT}{\mathsf{MT}}
\newcommand{\tSI}{\mathsf{SI}}
\newcommand{\SI}[3][]{\tSI_{#1}\,#2\,#3}
\newcommand{\tSIs}{\mathsf{SIS}}
\newcommand{\SIs}[3][]{\tSIs_{#1}\,#2\,#3}
\newcommand{\Labels}{\mathsf{Labels}}
\newcommand{\tLabel}{\mathsf{Label}}
\newcommand{\Label}[2][]{\tLabel_{#1}\,{#2}}
\newcommand{\tFC}{\mathsf{FC}}
\newcommand{\FC}[3][]{\tFC_{#1}\,#2\,#3}
\newcommand{\tlet}{\mathsf{let}}
\newcommand{\tfix}{\mathsf{fix}}
\newcommand{\texec}{\mathsf{exec}}
\newcommand{\vfc}{\mathit{fc}}
\newcommand{\tbranch}{\mathsf{branch}}
\newcommand{\tgoto}{\mathsf{goto}}
\newcommand{\Cmm}{\texttt{C--}\xspace}
\newcommand{\JVMmm}{\texttt{JVM--}\xspace}









\begin{document}

\maketitle
%\begin{frame}
%  \titlepage
%\end{frame}

%\begin{frame}
%  \frametitle{Outline}
%  \tableofcontents
%  % You might wish to add the option [pausesections]
%\end{frame}


% CompCert: Verified compilation of C
% Verification should be integral part of teaching compilation
% Full verification too expensive
% Type-soundness



\section{Introduction}


\begin{frame}%[fragile=singleslide]
  \frametitle{Verified Compilation}
  \vspace{-3ex}
  \begin{itemize}
  \item Tony Hoare's Grand Challenge: Verified compilation.
  \item CompCert for the masses?
% : First verified compiler for (subset of) C.
%   \item CompCert should not remain singular effort!
%   \item Can I teach verified compilation to my master students?
  \item Full verification may be too expensive ($> 90\%$ of impl.\ effort).
  \item Sweet spot: lots of confidence for little verification.
%  \item Low-hanging fruits?
  \item Compiler be a \alert{total} function.
  \end{itemize}
\end{frame}

% \begin{frame}%[fragile=singleslide]
%   \frametitle{Verified Compilation}
%   \vspace{-3ex}
%   \begin{itemize}
%   \item Tony Hoare's Grand Challenge: Verified compilation.
%   \item CompCert: First verified compiler for (subset of) C.
%   \item CompCert should not remain singular effort!
%   \item Can I teach verified compilation to my master students?
%   \item Full verification may be too expensive ($> 90\%$ of impl.\ effort).
%   \item Any low-hanging fruits?
%   \end{itemize}
% \end{frame}


\begin{frame}%[fragile=singleslide]
  \frametitle{Verifying Type-Safety}
  \vspace{-3ex}
  \begin{itemize}
  \item Robin Milner: \emph{Well-typed programs do not go wrong.}
  \item Types checked by compiler front-end.
  \item Goal: preserve properties through back-end.
    \begin{itemize}
    \item Type safety.
    \item ``Execution safety'': No illegal jumps.
    \end{itemize}
  \item Typed machine language (e.g. LLVM).
  \end{itemize}
\end{frame}


\begin{frame}%[fragile=singleslide]
  \frametitle{Method}
  \vspace{-3ex}
  \begin{itemize}
  \item Implement compiler in a dependently-typed programming
    language.
  \item Represent well-typed syntax as indexed data types.
  % \item Dependent types enable representation of well-scoped and
  %   well-typed syntax as indexed data types.
  \item Type-correct compilation enforced by indexing discipline.
%  \item Hope: Hard to write buggy yet type-preserving compiler.
  \item Employ algebraic structures (categories, functors, ...)
and algebraic presentations of graphs
$\longrightarrow$ \emph{systems modelling}.
  \end{itemize}
\end{frame}

\begin{frame}%[fragile=singleslide]
  \frametitle{Intrinsically well-typed syntax}
%  \vspace{-3ex}
  \begin{center}
  \begin{tabular}{|c|c|}%{|p30ex|p30ex|}
\hline
\textbf{object language} & \textbf{meta language} \\
\hline
%\vspace{1ex}
untyped & simply typed \\
\eg: syntax trees & \eg: (C, Java), Scala, ML, Haskell, \dots \\
%%\vspace{1ex}
\hline
%\vspace{1ex}
simply typed & dependently typed \\
\eg: $\lambda$-calculus, \Cmm & \eg: Agda, Coq, Idris, Lean, \dots \\
%%\vspace{1ex}
\hline
%\vspace{1ex}
dependently typed & \alert{dependently typed} \\
%%\vspace{1ex}
\hline
  \end{tabular}
  \end{center}
\end{frame}



\section{Compiler for \Cmm}

\begin{frame}%[fragile=singleslide]
  \frametitle{Pipeline}
  \vspace{-3ex}
  \begin{center}
\(
\xymatrix@R=6ex{
\text{\Cmm text}
\ar[d]^{\mybox{parser}}
\\
\text{abstract syntax}
\ar[d]^{\mybox{type checker}}
\\
\text{\alert{well-typed syntax}}
\ar[d]^{\text{\alert{code generator}}}
\\
\text{\alert{well-formed machine code}}
\ar[d]^{\mybox{printer}}
\\
\text{\JVMmm symbolic assembler}
}
\)
  \end{center}
\end{frame}


\begin{frame}[fragile=singleslide]
  \frametitle{\Cmm by example}
  \vspace{-3ex}
\begin{verbatim}
// Does p divide q?
bool divides (int p, int q) {
  return (q / p) * p == q;
}

// Is p prime?
bool prime (int p) {
  if (p <= 2) return p == 2;
  else {
    int q = 3;
    while (q * q <= p)
      if (divides(q,p)) return false;
      else q = q + 2;
  }
  return true;
}

\end{verbatim}
\end{frame}

\begin{frame}[fragile=singleslide]
  \frametitle{\Cmm language elements}
  \vspace{-3ex}
  \begin{itemize}
  \item Hierarchical:
    \begin{itemize}
    \item function defintions contain statements,
    \item statements contain expressions.
    \end{itemize}
  \item Types: $\Ty = \{ \tint, \tdouble,  \tbool, \tvoid \}$.
  \item Variables (function parameters, local variables) are \emph{scoped}.
  \item Some statements declare new variables (\verb|int q = 3;|).
  \item Control structures: \verb|if|, \verb|while|, \verb|return|.
  \end{itemize}
\end{frame}


\section{Well-typed syntax}

\begin{frame}%[fragile=singleslide]
  \frametitle{Typing contexts}
  \vspace{-3ex}
  \begin{itemize}
  \item Scoping is managed by \emph{typing contexts} $\Gamma$,
    snoc-lists of types.
  \item Example list $\tint^2 = [\tint,\tint]$:
$$
  \Ge.\tint.\tint
$$
\vspace{-2ex}
  \item Category $\Cxt$:
    \begin{itemize}
    \item Objects: typing contexts $\Gamma$.
    \item Morphisms $\GG \sublist \GD$ are ways in which $\GG$ is a
      sublist of $\GD$.
\[
  \nru{\tskip}{\vas \sublist \vbs}{\vas \sublist (\vbs.b)}
\qquad
  \nru{\tkeep}{\vas \sublist \vbs}{(\vas.a) \sublist (\vbs.a)}
\qquad
  \nru{\tdone}{}{\Ge \sublist \Ge}
\]
    \end{itemize}
  \item Variables (de Bruijn indexes) pick a type from a context.
\[
  \Var[t] \GG \cong ([t] \sublist \GG)
\]
\vspace{-2ex}
  \item Quiz:
    \begin{enumerate}
    \item How many morphisms in $\tint^2 \sublist \tint^5$?
    \item How many morphisms in $\tint^k \sublist \tint^n$?
    \end{enumerate}
  \end{itemize}
\end{frame}

\begin{frame}%[fragile=singleslide]
  \frametitle{$\Cxt$ has only weak push-outs}
%  \vspace{-3ex}
  \begin{center}
\(
\xymatrix@=7ex{
  [] \ar[rr] \ar[dd] && [a] \ar[dd] \ar[dddr] \\
  \\
  [b] \ar[rr] \ar[drrr] && [a,b] \\
  &&& [b,a] \\
}
\)
  \end{center}
\end{frame}

\begin{frame}%[fragile=singleslide]
  \frametitle{Well-typed syntax}
  \vspace{-3ex}
% Type checker produces well-typed syntax from parse trees.
  \begin{itemize}
  \item $\Var[t]\Gamma$ variables of type $t$
  \item $\Exp[t]\Gamma$ expressions of type $t$
  \item $\Stm[\vrt]\Gamma{\Gamma'}$ statements
    \begin{itemize}
    \item $\vrt$: return type of function
    \item $\Gamma$: context before statement
    \item $\Gamma' = \Gamma.\Delta$: context after
    \end{itemize}
  \item $\Stms[\vrt]{\Gamma}{\Gamma'}$ statement sequences: free
    category over $\tStm$.
  \end{itemize}
\end{frame}


\begin{frame}%[fragile=singleslide]
  \frametitle{Expressions}
  \vspace{-3ex}
  \begin{itemize}
  \item $\tExp_t : \Cxt \to \Set$ functor
  \item maps hom $\eta : \GG \sublist \GD$ to weakening
    $[\eta] : \Exp[t]\GG \to \Exp[t]\GD$
  \item constructors
\[
  \begin{array}{lcl@{~}c@{~}l}
\tlit   & : & (v : \Val[t]) & \to & \Exp[t]\Gamma \\
\tvar   & : & (x : \Var[t]\Gamma) & \to & \Exp[t]\Gamma \\
\tarith & : & (\vop : \ArithOp t)\ (e_1\;e_2 : \Exp[t]\Gamma) & \to &
              \Exp[t]\Gamma \\
% \tarith & : & (\vop : \ArithOp a)\ (e_1\;e_2 : \Exp[a]\Gamma) & \to &
%               \Exp[a]\Gamma \\
\tcmp   & : & (\vop : \CmpOp t)\ (e_1\;e_2 : \Exp[t]\Gamma) & \to &
              \Exp[\tbool]\Gamma \\
  \end{array}
\]
  \end{itemize}
\end{frame}


\begin{frame}%[fragile=singleslide]
  \frametitle{Statements}
  \vspace{-3ex}
%  \begin{itemize}
%  \item constructors of $\Stm[\vrt]{\Gamma}{\Gamma'}$
\[
\begin{array}{@{}l@{~}c@{~}l@{~}c@{~}l}
  \tassign & : & (x : \Var[t]\Gamma)\ (e : \Exp[t]\Gamma) & \to & \Stm[\vrt]\Gamma\Gamma \\
  \tdecl   & : & (t : \Ty) & \to & \Stm[\vrt]\Gamma{(\Gamma.t)} \\
  \treturn & : & (e : \Exp[\vrt]\Gamma) & \to & \Stm[\vrt]\Gamma\Gamma \\
  \twhile & : & (e : \Exp[\tbool]\Gamma)\
                 (s : \Stm[\vrt]\Gamma{\Gamma'}) & \to & \Stm[\vrt]\Gamma\Gamma \\
  \tif    & : & (e : \Exp[\tbool]\Gamma)\
                 (s_1 : \Stm[\vrt]\Gamma{\Gamma_1})\
                 (s_2 : \Stm[\vrt]\Gamma{\Gamma_2}) & \to & \Stm[\vrt]\Gamma\Gamma \\
\end{array}
\]
%  \end{itemize}
\end{frame}


\section{Well-formed machine code}

\begin{frame}%[fragile=singleslide]
  \frametitle{Java Virtual Machine (JVM)}
  \vspace{-3ex}
  \begin{itemize}
  \item no registers
  \item stack for evaluating expressions
  \item local variable store (incl. function parameters)
  \item (heap for objects)
  \item method call handling behind the scenes
  \end{itemize}
\end{frame}

\begin{frame}[fragile]%=singleslide]
  \frametitle{Java Virtual Machine (JVM) example}
  \vspace{-1ex}
  \begin{tabular}{l@{\qquad}l}
  \Cmm & Jasmin (symbolic JVM) \\
\hline
\\[-1ex]
    \begin{minipage}[t]{0.3\linewidth}
\begin{verbatim}
bool divides
   (int p, int q)
{
  return
  (q / p) * p == q;
}
\end{verbatim}
    \end{minipage}
& % translates to Jasmin (symbolic JVM) code:
  \begin{minipage}[t]{0.5\linewidth}
\begin{verbatim}
.method divides(II)I
        iload_1           ;; q
        iload_0           ;; p
        idiv
        iload_0           ;; p
        imul

        iload_1           ;; q
        if_icmpeq L_true
        iconst_0          ;; false
        goto L_done
L_true: iconst_1          ;; true
L_done: ireturn
.end method
\end{verbatim}
  \end{minipage}
  \end{tabular}
\end{frame}


\begin{frame}%[fragile=singleslide]
  \frametitle{Evaluation Stack}
  \vspace{-3ex}
  \begin{itemize}
  \item JVM has local stack for evaluation of expressions.
  \item Stack type $\ST = \List\,\Ty$
  \item Stack instruction $\SI[\Gamma]\Phi{\Phi'}$
    \begin{itemize}
    \item $\Gamma : \Cxt$ local variable store typing
    \item $\Phi   : \ST$ stack typing before instruction
    \item $\Phi'  : \ST$ stack typing after
    \end{itemize}
  \item Constructors:
\[
\begin{array}{lcl@{~}c@{~}l@{}l}
  \tldc & : & (i : \Val[\tint]\Gamma) & \to & \SI[\Gamma]\Phi{&(\Phi.\tint)} \\
  \tload & : & (x : \Var[t]\Gamma) & \to & \SI[\Gamma]\Phi{&(\Phi.t)} \\
  \tstore & : & (x : \Var[t]\Gamma) & \to & \SI[\Gamma]{(\Phi.t)}{&\Phi} \\
  \tarith & : & (\vop : \ArithOp t) & \to & \SI[\Gamma]{(\Phi.t.t)}{&(\Phi.t)}
\end{array}
\]
  \item Instruction sequences $\SIs[\Gamma]\Phi{\Phi'}$: free category over $\tSI_\Gamma$.
  \end{itemize}
\end{frame}


\begin{frame}%[fragile=singleslide]
  \frametitle{Variable typing administration}
  \vspace{-3ex}
  \begin{itemize}
  \item Variable declarations $\tdecl\,t : \Stm \Gamma {(\Gamma.t)}$ are NOPs.
%  \item Only needed for typing.
  \item Needed in intrinsically typed machine language.
\[
  \tdecl\,t : {(\Gamma,\Phi)} \to {(\Gamma.t,\Phi)}
\]
\vspace{-2ex}
  \item Reconstruction in actual JVM by static analysis (bytecode verifier).
  \item Machine type $\MT = \Cxt \times \ST$.
  \end{itemize}
\end{frame}


\begin{frame}[fragile=singleslide]
  \frametitle{Jumps can go wrong}
  \vspace{-3ex}
  \begin{itemize}
  \item Bad jump:
\begin{verbatim}
                          ;; Stack modification:
        if_icmpeq L_true  ;; [int,int] -> []
        iconst_0          ;; []        -> [int]
L_true: istore_3          ;; [int]     -> []

\end{verbatim}
  \item Jump target needs to have same stack typing as source.
  \item Same for variable typing.
  \item Labels are typed by ``before'' machine type $\Xi$ of target.
  \item Label context $\Labels = \List\,\MT$.
  \item A label is a de Bruijn index $\ell : \Label[\Xi] \GL$.
\[
  \Label[\Xi] \GL \cong ([\Xi] \sublist \GL)
\]
  \end{itemize}
\end{frame}


\begin{frame}[fragile=singleslide]
  \frametitle{Jump targets need to exist}
  \vspace{-3ex}
  \begin{itemize}
  \item Semantics of a label is the code following it.
  \item Each label needs to point to some code.
  \item Two types of labels:
    \begin{itemize}
    \item \emph{Join points} for branches of $\tif$ are $\tlet$s.
% \begin{verbatim}
% [[ if (e) s1; else s2; s ]] =
%   let l  = [[s]]
%       l1 = [[s1]]; goto l
%       l2 = [[s2]]; goto l
%   in  [[e]]; branch l1 l2
% \end{verbatim}
    \item \emph{Back jumps} to repeat body of $\twhile$ are $\tfix$s.
% \begin{verbatim}
% [[ while (e) s0; s ]] =
%   let l2 = [[s]]
%   in  fix l.
%         let l1 = [[s0]]; goto l
%         in  [[e]]; branch l1 l2
% \end{verbatim}
    \end{itemize}
  \end{itemize}
\end{frame}

\begin{frame}[fragile=singleslide]
  \frametitle{Join points: let}
\begin{verbatim}
[[ if (e) s1; else s2; s ]] =

  let l  = [[s]]
      l1 = [[s1]]; goto l
      l2 = [[s2]]; goto l
  in  [[e]]; branch l1 l2
\end{verbatim}
\end{frame}


\begin{frame}[fragile=singleslide]
  \frametitle{Back jumps: fix}
\begin{verbatim}
[[ while (e) s0; s ]] =

  let l2 = [[s]]
  in  fix l.
        let l1 = [[s0]]; goto l
        in  [[e]]; branch l1 l2
\end{verbatim}
\end{frame}

\begin{frame}%[fragile=singleslide]
  \frametitle{Flowchart (control flow graph)}
  \vspace{-3ex}
  \begin{itemize}
  \item $\FC[\vrt] \Xi \GL$ control flow graph
    \begin{itemize}
    \item $\vrt$ return type of method
    \item $\Xi$ machine state on entry
    \item $\GL$ typed jump targets
    \end{itemize}
  \item Constructors:
\[
\begin{array}{lcl@{}l@{~}c@{~}l@{}l}
  \texec   & : & (i : \SI[\Gamma] \Phi {\Phi'})\ & (\vfc : \FC[\vrt] {(\Gamma,\Phi')} \GL)
                 & \to & \FC[\vrt] {(\Gamma,\Phi)} {&\GL} \\
  \tdecl   & : & (t : \Ty)\ & (\vfc : \FC[\vrt] {(\Gamma.t,\Ge)} \GL) & \to & \FC[\vrt] {(\Gamma,\Ge)} {&\GL} \\
  % \tdecl   & : & (t : \Ty)\ & (\vfc : \FC[\vrt] {(\Gamma.t,\Phi)} \GL) & \to & \FC[\vrt] {(\Gamma,\Phi)} {&\GL} \\
  \treturn & : & (e : \Exp[\vrt] \Gamma) & & \to & \FC[\vrt] {(\Gamma,\Ge)} {&\GL} \\
  \tgoto   & : & (\ell : \Label[\Xi]\Gamma) & & \to & \FC[\vrt] \Xi {&\GL} \\
  \tbranch & : & (\vop : \CmpOp t) \ & (\vfc_1\;\vfc_2 : \FC[\vrt] {(\Gamma,\Phi)} \GL) & \to & \FC[\vrt] {(\Gamma,\Phi.t.t)} {&\GL} \\
  \tlet    & : & (\vfc' : \FC[\vrt] {\Xi'} \GL)\ & (\vfc : \FC[\vrt] \Xi {(\GL.\Xi')}) & \to & \FC[\vrt] \Xi {&\GL} \\
  \tfix    & : & & (\vfc : \FC[\vrt] \Xi {(\GL.\Xi)}) & \to & \FC[\vrt] \Xi {&\GL} \\
\end{array}
\]
  \end{itemize}
\end{frame}


\section{Wrapping up}


\begin{frame}%[fragile=singleslide]
  \frametitle{Back end}
  \vspace{-3ex}
  \begin{itemize}
  \item Code generation: translation from well-typed syntax to flow chart using continuations.
  \item Linearization: from flowcharts to basic blocks.
  \item Printing: from basic blocks to Jasmin symbolic JVM.
  \end{itemize}
\end{frame}

\begin{frame}%[fragile=singleslide]
  \frametitle{Evaluation}
  \vspace{-3ex}
  \begin{itemize}
  \item \emph{When it type-checks, it works.}
  \item Had only 3 bugs in compiler on first run!
  \item Agda programming requires hard thinking ahead.
  \item Little proof effort.
  \item Too hard for average beginning master student.
%  \item Agda not yet for the masses.
%  \item Fine for a master thesis (Alexander Fuhs).
  \item Full verification in progress:
    \begin{itemize}
    \item Needs reasoning in sublist-category.
    \item Contributed categorical constructions (e.g. weak pushout) to Agda standard library.
    \end{itemize}
  \end{itemize}
\end{frame}


% \begin{frame}%[fragile=singleslide]
%   \frametitle{}
%   \vspace{-3ex}
%   \begin{itemize}
%   \item
%   \end{itemize}
% \end{frame}


% \begin{frame}[fragile=singleslide]
%   \frametitle{Conclusions}
%   \vspace{-3ex}
%   \begin{itemize}
%   \item
%   \end{itemize}
% \end{frame}

\begin{frame}%[fragile=singleslide]
  \frametitle{Related Work}
  %\vspace{-3ex}
  \begin{itemize}
  \item Andrew Appell, Modern compiler implementation in C/Java/ML
  \item Xavier Leroy et al., CompCert, in Coq
  \item Magnus Myreen et al., CakeML, in HOL
  \item Alberto Pardo, Emmanuel Gunter, Miguel Pagano, Marcos Viera,
\emph{An Internalist Approach to Correct-by-Construction Compilers},  PPDP'18:
Terms indexed by semantics (in Agda)
  \item DeepSpec project: Verified tool chain.
  \end{itemize}
\end{frame}


% \begin{frame}%[fragile=singleslide]
%   \frametitle{Future work}
%   %\vspace{-3ex}
%   \begin{itemize}
%   \item
%   \end{itemize}
% \end{frame}


%%%%%%%%%%%%%%%%%%%%%%%%%%%%%%%%%%%%%%%%%%%%%%%%%%%%%%%%%%%%%%%%%%%%%%
%%%%%%%%%%%%%%%%%%%%%%%%%%%%%% END DOC %%%%%%%%%%%%%%%%%%%%%%%%%%%%%%%
%%%%%%%%%%%%%%%%%%%%%%%%%%%%%%%%%%%%%%%%%%%%%%%%%%%%%%%%%%%%%%%%%%%%%%

% \bibliography{short}

\end{document}




\begin{frame}%[fragile=singleslide]
  \frametitle{}
  \vspace{-3ex}
  \begin{itemize}
  \item
  \end{itemize}
\end{frame}


\begin{frame}%[fragile=singleslide]
  \frametitle{}
  \vspace{-3ex}
  \begin{itemize}
  \item
  \end{itemize}
\end{frame}


\begin{frame}%[fragile=singleslide]
  \frametitle{}
  \vspace{-3ex}
  \begin{itemize}
  \item
  \end{itemize}
\end{frame}


\begin{frame}%[fragile=singleslide]
  \frametitle{}
  \vspace{-3ex}
  \begin{itemize}
  \item
  \end{itemize}
\end{frame}


\begin{frame}%[fragile=singleslide]
  \frametitle{}
  \vspace{-3ex}
  \begin{itemize}
  \item
  \end{itemize}
\end{frame}


\begin{frame}%[fragile=singleslide]
  \frametitle{}
  \vspace{-3ex}
  \begin{itemize}
  \item
  \end{itemize}
\end{frame}


\begin{frame}%[fragile=singleslide]
  \frametitle{}
  \vspace{-3ex}
  \begin{itemize}
  \item
  \end{itemize}
\end{frame}

\end{document}




%%% Local Variables:
%%% mode: latex
%%% TeX-master: t
%%% End:
