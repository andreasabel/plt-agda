\nonstopmode
\RequirePackage[dvipsnames]{xcolor}

\documentclass[a4paper]{easychair}
\usepackage{amssymb}
%\usepackage{dsmath}
%\usepackage{bbm}
\usepackage{bbold}
\usepackage[utf8x]{inputenc}

\usepackage{natbib}
\usepackage{hyperref}
\urlstyle{sf} % less wide than tt
% \usepackage{agda} % colors

\definecolor{DeepJungleGreen}{rgb}{0.0, 0.29, 0.29}
\definecolor{DarkSpringGreen}{rgb}{0.09, 0.45, 0.27}
\definecolor{ForestGreenTraditional}{rgb}{0.0, 0.27, 0.13}
\definecolor{GreenPigment}{rgb}{0.0, 0.65, 0.31}
\definecolor{CalPolyPomonaGreen}{rgb}{0.12, 0.3, 0.17}
\definecolor{CSSGreen}{rgb}{0.0, 0.5, 0.0}
\definecolor{DarkCerulean}{rgb}{0.03, 0.27, 0.49}
\definecolor{DarkRaspberry}{rgb}{0.53, 0.15, 0.34}

\colorlet{DarkBlue}{MidnightBlue!80!black}
\definecolor{HanBlue}{rgb}{0.27, 0.42, 0.81}
\definecolor{FrenchBlue}{rgb}{0.0, 0.45, 0.73}
\definecolor{EgyptianBlue}{rgb}{0.06, 0.2, 0.65}

\hypersetup{colorlinks=true,
  linkcolor=Red,
  citecolor=DarkRaspberry,%DarkPurple,%DarkCerulean,%CSSGreen,%CalPolyPomonaGreen,%GreenPigment,%ForestGreenTraditional,%DarkSpringGreen,%DeepJungleGreen,%ForestGreen,
  urlcolor=MidnightBlue}

\makeatletter
\renewcommand\bibsection{%
  \section*{\large{\refname}}%
}%
\makeatother

% less space between entries in bibliography
\let\OLDthebibliography\thebibliography
\renewcommand\thebibliography[1]{
  \OLDthebibliography{#1}
  \setlength{\parskip}{0pt}
  \setlength{\itemsep}{0.2ex plus 0.3ex}
}

%% Macros

\input{macros}

\renewcommand{\patom}{\ensuremath{o}}
\renewcommand{\tVnf}{\mathsf{Nf}}

\begin{document}

\title{Type-preserving compilation via dependently typed syntax}% in Agda}
\titlerunning{Type-preserving compilation via dependently typed syntax}
\authorrunning{A. Abel}
\author{Andreas Abel\thanks{Supported by VR grants 621-2014-4864 and
    2019-04216 and EU
  Cost Action CA15123.}}
\institute{
  Department of Computer Science and Engineering,
  Gothenburg University
}

% Keywords:
% Agda, Dependent types, Indexed data types, Typed Assembly, Verified compilation

\maketitle

%\begin{abstract}
%\end{abstract}

\noindent

The \emph{CompCert} project \citep{leroy:cacm09} produced a verified
compiler for a large fragment of the C programming language.  The
CompCert compiler is implemented in the type-theoretic proof assistant
Coq \citep{inria:coq89}, and is fully verified: there is a proof that
the semantics of the source program matches the semantics of the
target program.  However, full verification comes with a price: the
majority of the formalization is concerned not with the runnable code
of the compiler, but with properties of its components and proofs of
these properties.
If we are \emph{not} willing to pay the price of full verification, can we
nevertheless profit from the technology of type-theoretic proof
assistants to make our compilers \emph{safer} and \emph{less likely}
to contain bugs?

In this talk, I am presenting a compiler for a small fragment of the C
language using \emph{dependently-typed syntax}
\citep{bentonHurKennedyMcBride:jar12,allaisAtkeyChapmanMcBrideMcKinna:icfp18}.
A typical compiler is proceeding in phases: parsing, type checking,
code generation, and finally, object/binary file creation.  Parsing and type
checking make up the \emph{front end}, which may report syntax and
type errors to the user; the other phases constitute the \emph{back
  end} that should only fail in exceptional cases.  After type
checking succeeded, we have to deal only with well-typed source
programs, whose abstract syntax trees can be captured with the indexed
data types of dependently-typed proof assistants and programming
languages like Agda \citep{agda:260}, Coq, Idris \citep{brady:jfp13},
Lean \citep{deMoura:cade15} etc.  More concisely, we shall by
\emph{dependently-typed syntax} refer to the technique of capturing
well-typedness invariants of syntax trees.

Representing also typed assembly language
\citep{morrisettWalkerCraryGlew:toplas99} via dependently-typed
syntax, we can write a type-preserving compiler whose type soundness
is given \emph{by construction}.  In the talk, the target of
compilation is a fragment of the Java Virtual Machine (JVM) enriched
by some \emph{administrative instructions} that declare the types of
local variables.  With JVM being a stack machine, instructions are
indexed not only by the types of the local variables, but also by the
types of the stack entries before and after the instruction.  However
for instructions that change the control flow, such as unconditional
and conditional jumps, we need an additional structure to ensure type
safety.  Jumps are safe if the jump target has the same machine typing
than the jump source.  By \emph{machine typing} we mean the pair of
the types of the local variables and the types of the stack entries.
Consequently, each label (\ie, jump target) needs to be assigned a
machine type and can only be targeted from a program point with the
same machine type.  Technically, we represent labels as machine-typed
de Bruijn indices, and control-flow instructions are indexed by a
context of label types.  We then distinguish two types of labels:
\begin{enumerate}
\item Join points, \eg, labels of statements following an \texttt{if-else} statement.
  Join points can be represented by a \texttt{let} binding in the abstract JVM
  syntax.
\item Looping points, \eg, labels at the beginning of a \texttt{while}
  statement that allow back jumps to iterate the loop.  Those are
  represented by \texttt{fix} (recursion).
\end{enumerate}
Using dependently-typed machine syntax, we ensure that
\emph{well-typed jumps do not miss}.  As a result, we obtain a
type-preserving compiler by construction, with a good chance of full
correctness, since many compiler faults already break typing invariants.
Intrinsic well-typedness also allows us to write the compiler as a
total function from well-typed source to typed assembly, and totality
can be automatically verified by the Agda type and termination checker.


% \paragraph*{Acknowledgments.}
% %Thanks to the anonymous referees, who helped improving the quality of
% %this abstract through their feedback.
% This work was supported by
% Vetenskapsr\aa{}det under Grant
%   No.~621-2014-4864 \emph{Termination Certificates
%   for Dependently-Typed Programs and Proofs via Refinement Types}
% and by the EU Cost Action CA15123 \emph{Types for programming and verification}.

%\begin{footnotesize}

\bibliographystyle{abbrvnat}
%\bibliographystyle{abbrvurl}
%\bibliographystyle{plainurl} % no author-year cit.
\bibliography{auto-types20}

%\end{footnotesize}
\end{document}
